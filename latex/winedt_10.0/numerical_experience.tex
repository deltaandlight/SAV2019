\documentclass{beamer}
\usepackage{amsmath}
\usepackage{amsfonts}
\usepackage{amssymb}
\usepackage{graphicx}
\usepackage{subfigure}

\begin{document}
\begin{frame}
	\title{Numerical experiments of \\
scalar auxiliary variable (SAV) approach}
	\author{Liang Zeng}
    \date{\today}
	\institute {Email: zengliangpku@pku.edu.cn, \\
                School of Mathematical Sciences, Peking University}
	
\titlepage
\end{frame}

\begin{frame}
\frametitle{Example of Cahn-Hilliard equation}
The free energy of Cahn-Hilliard equation takes the form,
$$
\mathrm{E}(\phi)=
\int_{\Omega}
\Big\{
\frac{1}{2}{\left|\nabla\phi\right|}^2
+\frac{({\phi}^2-1)^2}{4{\epsilon}^2}
\Big\}dx
$$

For the SAV schemes, we specify the linear non-negative operator as $\mathrm{L}=-\Delta$ and the non-positive operator as $\mathrm{G}=\Delta$.

The CH equation takes the form,
\begin{equation}\label{CN}
\frac{\partial\phi}{\partial t}
=\mathrm{G}(\mathrm{L}\phi
+\frac{({\phi}^2-1)^2}{{\epsilon}^2})
\end{equation}
\end{frame}

\begin{frame}
\frametitle{Discrete form}
The SAV/BDF3 scheme is given by
\begin{equation}\label{BDF3}
\begin{aligned}
   \frac{11{\phi}^{n+1}-18{\phi}^n + 9{\phi}^{n-1}-2{\phi}^{n-2}}{6\Delta t} = \mathrm{G}{\mu}^{n+1}&, \\
   {\mu}^{n+1}  = \mathrm{L}{\phi}^{n+1} 
   + \frac{r^{n+1}}{\sqrt{\mathrm{E}_1\left[{\overline{\phi}}^{n+1}\right]}}&, \\
    11r^{n+1}-18r^{n}+r^{n-1}-2r^{n-2} &= \\
    \int_{\Omega}
    \frac{U\left[{\overline{\phi}}^{n+1}\right]}{2\sqrt{\mathrm{E}_1\left[{\overline{\phi}}^{n+1}\right]}}
    (11{\phi}^{n+1}-18{\phi}^n+9{\phi}^{n-1}&-2{\phi}^{n-2})dx,
\end{aligned}
\end{equation}
where ${\overline{\phi}}^{n+1}$ is a third-order explicit approximation to $\phi(t_{n+1})$.
\end{frame}

\begin{frame}
\frametitle{Numerical result}
We fix the computational domain as $\left[0,2\pi\right)^2$ and $\epsilon = 0.1$. $\Delta T={10}^{-5}$ and the size of uniform grid is $N\times N=2^7 \times 2^7$. The initial data is $u_0(x,y)=0.05\sin(x)\sin(y)$.

The energy evolution process are shown in Fig.\ref{CH.T0.032}, the process ends when $T=0.032$.
\begin{figure}[h]
    \centering
    \begin{minipage}[t]{0.5\linewidth}
        \centering
        \includegraphics[width=1.8in]{pic/example6_fig6/dt=1e-5_endT=0032.jpg}
        \caption{Simulated solution}
        \label{CH.T0.032}
    \end{minipage}%
    \begin{minipage}[t]{0.5\linewidth}
        \centering
        \includegraphics[width=1.8in]{pic/example6_fig6/0REF_dt=1e-4_endT=0032.jpg}
        \caption{Reference solution}
        \label{REF_CH.T0.032}
    \end{minipage}   
\end{figure}

The numerical result is consistent with the reference solution.
\end{frame}

\begin{frame}
\frametitle{Example of Phase field crystals}
A usual free energy takes the form,
$$
\mathrm{E}(\phi)=
\int_{\Omega}
\Big\{
\frac{1}{4}\phi
+\frac{1-\epsilon}{2}{\phi}^2
-{\left|\nabla\phi\right|}^2
+\frac{1}{2}{(\Delta\phi)}^2
\Big\}dx.
$$

For the SAV schemes, we specify the linear non-negative operator as $\mathrm{L}={\Delta}^2+2\Delta+I$ and the non-positive operator as 
$\mathrm{G}=\Delta$. The equation takes the form,

\begin{equation}\label{PFC}
\frac{\partial\phi}{\partial t}
=\mathrm{G}(\mathrm{L}\phi
-\epsilon\phi
+{\phi}^3).
\end{equation}

\end{frame}

\begin{frame}
\frametitle{Numerical result}
We consider the gradient flow equation in the two-dimensional domain $\left[0,50\right]\times\left[0,50\right]$ with periodic boundary conditions. Fix $\epsilon = 0.025, \Delta T=1$ and the size of uniform grid $N\times N=2^7 \times 2^7$.
 
The initial values $\phi_0$ are uniformly distributed from 0.01 to 0.14 with $\overline{\phi_0}=0.07$. We also use BDF3A discrete form to simulate the process. When $T=4800$, $\phi$ becomes the result in Fig.\ref{PFC.T4800}, which is consisitent with what is known.

\begin{figure}[h]
  \centering
  \includegraphics[width=5cm]{pic/example5_fig4/myfig_image48.jpg}
  \caption{$T=4800$}\label{PFC.T4800}
\end{figure}
\end{frame}

\end{document} 